\documentclass[a4paper, 12pt]{article}

\usepackage[table]{xcolor}


\begin{document}
Hier werden die unterschiedlichen Arten und Erstellungsmethoden von Tabellen dargestellt.

So sieht die einfachste Tabelle aus:

\begin{tabular}{|c|c|c|}
\hline
Spalte 1 & Spalte 2 & Spalte 3 \\
\hline
1 & 2 & 3 \\
\hline
\end{tabular}

Und wenn man eine Überschrift haben will:

\begin{tabular}{|c|c|c|}
\hline
\multicolumn{3}{|c|}{Überschrift}\\
\hline
Spalte 1 & Spalte 2 & Spalte 3 \\
\hline
1 & 2 & 3 \\
\hline
\end{tabular}

Es gibt unterschiedliche Zelleneinrichtungen möglich:

\begin{tabular}{|l|c|r|}
    \hline
    Linksbündig & Zentriert & Rechtsbündig \\
    \hline
    Apfel & Banane & Kirsche \\
    \hline
    Hund & Katze & Maus \\
    \hline
\end{tabular}

Tabelle mit Farben

\begin{tabular}{|c|c|c|}
\hline
\rowcolor{blue!25}
Spalte 1 & Spalte 2 & Spalte 3 \\
\hline
1 & 2 & 3 \\
\hline
\end{tabular}

\end{document}